\documentclass[11pt, oneside]{article}   	% use "amsart" instead of "article" for AMSLaTeX format
\usepackage{geometry}                		% See geometry.pdf to learn the layout options. There are lots.
\geometry{letterpaper}                   		% ... or a4paper or a5paper or ... 
%\geometry{landscape}                		% Activate for rotated page geometry
%\usepackage[parfill]{parskip}    		% Activate to begin paragraphs with an empty line rather than an indent
\usepackage{graphicx}				% Use pdf, png, jpg, or eps with pdflatex; use eps in DVI mode
								% TeX will automatically convert eps --> pdf in pdflatex		
\usepackage{amsmath}
\usepackage{mathtools}
\usepackage{amssymb}        % so we can use the 'pretty' empty set
\usepackage{tipa}
\usepackage{graphicx}           % purdy pitchers

\title{An algorithm to assign features to a set of natural classes}
\author{}
\author{
  Mayer, Connor Joseph \\
  \texttt{connor.joseph.mayer@gmail.com}
  \and
  Daland, Robert \\
  \texttt{r.daland@gmail.com}
}
%\date{}							% Activate to display a given date or no date

\begin{document}
\maketitle

\begin{abstract}
This squib describes a dynamic programming algorithm which assigns features to a set of natural classes. The input consists of a set of classes, each containing one or more segments; in other words, a subset of the powerset of a segmental alphabet $\Sigma$. If a class can be generated as the union of existing features ( = intersection of already-processed classes), those features are propagated to every segment in the class. Otherwise, a new feature/value is assigned. The algorithm comes in 4 flavors, which differ with respect to complementation and how negative values are assigned. We show that these variants yield \textit{privative specification}, \textit{contrastive underspecification}, \textit{contrastive specification}, and \textit{full specification}, respectively. The main text sets out necessary background, and illustrates each variant of the algorithm. The Appendix formally proves that each algorithm is sound.
\end{abstract}

\section{Introduction}
merge what Connor wrote

\section{Definitions and notation}

Let $\Sigma$ denote an alphabet of segments. We will use the term \textit{class} to mean a subset of $\Sigma$. 

\subsection{Definition and example of natural class system}

% definition of natural class system
A \textit{natural class system} $\mathcal C$ is a set of classes over $\Sigma$, $\mathcal C = \{C_i\}_{i=1}^N$, which includes $\Sigma$ itself, and the empty set (i.e. $\varnothing , \Sigma \in \mathcal C$).

Readers who are familiar with the notion of \textit{lattice} will note that every natural class system forms a lattice under the subset relation. To illustrate, a vowel harmony lattice is shown below (the empty set is suppressed):

% show an example of a natural class system: a vowel harmony lattice
\begin{figure}[h]
\includegraphics[width=0.9\textwidth]{vowelHarmony_unicode.png}
\caption{Vowel harmony lattice}
\label{fig:lattice}
\end{figure}

\subsection{Definition and example of a feature system}

% definition of feature system
A \textit{feature system} is a tuple $(\mathcal F, \Sigma, \mathcal V)$ where \begin{itemize}
    \item $\Sigma$ is a segmental alphabet, 
    \item $\mathcal V$ is a set of values, and 
    \item $\mathcal F = \{f_j : \Sigma \rightarrow \mathcal V\}_{j=1}^M$ is a set of feature functions mapping segments to feature values
    \end{itemize}

We say that a feature system has \textit{privative specification} if $\mathcal V = \{ +, 0 \}$, \textit{full specification} if $\mathcal V = \{ +, - \}$, and \textit{contrastive specification} if $\mathcal V = \{ +, -, 0 \}$. We do not consider other value sets here.
        
A (fully specified) feature system for the vowel harmony lattice shown in Fig.~\ref{fig:lattice} is shown below:

% table with featurization of vowel harmony lattice
\begin{table}[h]
    \centering
    \begin{tabular} {|c||c|c|c|c|c|}
    \hline
        $\sigma$ & front & back & low & high & round \\ \hline
        \textipa{i} & + & -- & -- & + & -- \\
        \textipa{y} & + & -- & -- & + & + \\
        \textipa{W} & -- & + & -- & + & -- \\
        \textipa{u} & -- & + & -- & + & + \\
        \textipa{E} & + & -- & -- & -- & -- \\
        \textipa{\oe} & + & -- & -- & -- & + \\
        \textipa{2} & -- & + & -- & -- & -- \\
        \textipa{O} & -- & + & -- & -- & + \\
        \textipa{a} & -- & + & + & -- & -- \\
        \hline
    \end{tabular}
    \caption{Example of a (fully specified) featurization.}
    \label{table:featurization}
\end{table}

\subsection{Featural descriptors}

\vspace{\baselineskip} \noindent Let $(\mathcal F, \Sigma, \mathcal V)$ be a feature system. The following definitions will prove useful: \begin{itemize}
    \item We will refer to the set of feature functions $\mathcal F = \{f_j\}_{j=1}^M$ as a \textit{featurization} (of $\Sigma$). 
    \item A \textit{featural descriptor} $\mathbf{e}$ is a subset of $\mathcal (V \setminus \{0\}) \times \mathcal F$, in other words a set of feature/value pairs where the value cannot be $0$ \begin{itemize}
        \item featural descriptors can be written in the form $[\alpha_k f_k]_{k \in K}$ for some index set $K$
        \item an example is $[+ \text{front}, - \text{low}]$
        \end{itemize}
    \item The natural class described by a featural descriptor $\mathbf{e}$, written $\langle \mathbf{e} \rangle$, consists of every segment which has \textit{at least} the feature/value pairs in $\mathbf{e}$ \begin{itemize}
        \item $\mathbf{e} = [\alpha_k f_k]_{k \in K} \leftrightarrow \langle \mathbf{e} \rangle = \{x \in \Sigma \, | \, \forall k \in K \, [ f_k(x) = \alpha_k ] \}$
        \item for the feature system in Table~\ref{table:featurization}, the natural class described by $[+ \text{front}, - \text{low}]$ is \{ \textipa{i}, \textipa{y}, \textipa{E}, \textipa{\oe} \}
        \end{itemize}
    \item Let $\mathcal V^\mathcal F$ denote the set of all licit featural descriptors over $(\mathcal F, \Sigma, \mathcal V)$. Let $\langle \mathcal V^\mathcal F \rangle$ indicate the set of natural classes described by this set. We say that the feature system $(\mathcal F, \Sigma, \mathcal V)$ generates $\langle \mathcal V^\mathcal F \rangle$.
    \end {itemize}
    
Note that while every featural descriptor in $\mathcal V^\mathcal F$ picks out a class in $\langle \mathcal V^\mathcal F \rangle$, the two are not in 1-1 correspondence. This is because the same class can often be described by multiple featural descriptors. For example, under the the vowel feature system shown in Table~\ref{table:featurization}, $[+front, -front]$ and $[+high, +low]$ both pick out the empty set.

\subsection{Properties of feature systems}

\vspace{\baselineskip} \noindent Let $(\mathcal F, \Sigma, \mathcal V)$ be a feature system with featurization $\mathcal F = \{f_j\}_{j=1}^M$. \begin{itemize}
    \item The \textit{feature vector} of a segment $x$ is the tuple $F(x) = (f_j(x))_{j=1}^k$.
    \item Two segments $x, y$ are \textit{featurally distinct} if and only if $F(x) \neq F(y)$; in other words, if they do not match on at least feature.
    \item The feature system is \textit{well-formed} if every pair of segments in $\Sigma$ is featurally distinct.\footnote{It is always possible to make ill-formed systems become well-formed. For example, suppose that [ptk] are not given distinct place features. One way to make the system well-formed is to add place features. Another way is to replace instances of [ptk] with a meta-symbol [T] in $\Sigma$, yielding a new segmental alphabet $\Sigma ' = \Sigma \setminus \{p,t,k\} \cup \{T\}$.}
    \item Let $\mathcal C = \{C_i\}_{i=1}^N$ be a set of natural classes. A featurization is \textit{adequate} to distinguish $\mathcal C$ if and only if $\mathcal C \subset \langle \mathcal V^\mathcal F \rangle$, in other words if it provides a way to pick out every class in $\mathcal C$.
    \item A feature $f_j$ is \textit{redundant} if $\mathcal F' = \mathcal F \setminus \{ f_j \}$ is well-formed.
    \item A featurization is \textit{efficient} if it contains no redundant features.
    \item A featurization is \textit{minimal} (for the value set $\mathcal V$) if there is no well-formed featurization that contains a smaller number of feature functions than $\mathcal F$ does (mapping to the same value set).
    \end{itemize}
    
\vspace{\baselineskip} \noindent It is straightforward to show that if $\mathcal F(\Sigma)$ is a well-formed featurization, then it generates a natural class system. Our goal in the remainder of this paper is the following: \begin{itemize}
    \item Suppose that the learner has evidence (based on phonetic similarity, and/or phonological patterning) for a set of natural classes $\mathcal C = \{C_i\}_{i=1}^N$
    \item Describe an algorithm which returns a well-formed feature system that is \textit{adequate} for $\mathcal C$, \textit{efficient}, and ideally \textit{minimal}.
    \end{itemize}

\section{A dynamic programming algorithm for computing intersectional closure}

Let $\mathcal C = \{C_i\}_{i=1}^n$ be a natural class system. The \textit{intersective closure} of $\mathcal C$, denoted $\mathcal C_\cap$, is the natural class system consisting of every class that can be generated by the intersection of finitely many classes in $\mathcal C$ (e.g. every class in $\mathcal C$, as well as any class that can be generated by the intersection of two or more classes in $\mathcal C$). Formally, 

\section{Privative specification}
achieved by assigning a new feature [+f] only, to every segment in $X$

\section{Contrastive underspecification}
achieved by assigning a new feature [+f] to every segment in $X$, and if $Y \setminus X$ (the complement of $X$ with respect to $Y$) is in the input, then [-f] is assigned to every segment in $Y \setminus X$

\section{Contrastive specification}
achieved by assigning a new feature [+f] to every segment in $X$, and [-f] to every segment in $Y \setminus X$ (even if $Y \setminus X$ was not in the input)

\section{Full specification}
achieved by assigning a new feature [+f] to every segment in $X$, and [-f] to every segment in $\Sigma \setminus X$

\appendix

\section{Formal proof of the algorithm}

\subsection{Privative underspecification}

\subsection{Contrastive underspecification}

\subsection{Contrastive specification}

\subsection{Full specification}

\end{document}  